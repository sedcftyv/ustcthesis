% !TeX root = ../main.tex

\chapter{绪论}

\section{研究背景}
纹理(Texture)是三维渲染过程中使用的图像数据,该图像的各个像素称为纹素。如图\ref{fig:TextureMapping}所示,最简单的纹理是映射到三维表面的二维图像。
纹理不仅可以存储颜色,还可以存储高度、法线方向、材质属性(金属度、粗糙度)以及光照等信息。
如图\ref{fig:TextureMapping-sample}所示,A、B和C分别为漫反射纹理、法线纹理与高度纹理,漫反射纹理决定物体在标准光照条件下的外观、法线纹理为其增添阴影的视觉效果、高度纹理用于模拟表面凹凸的效果。

\begin{figure}[htbp]
    \centering
    \includegraphics[width=\textwidth]{figures/TextureMapping.png}
    \caption{纹理映射示例(三角形的相邻并不意味着相应纹理区域的相邻)\cite{paltashev2014texture}}
    \label{fig:TextureMapping}
\end{figure}

\begin{figure}[htbp]
    \centering
    \includegraphics[width=\textwidth]{figures/TextureMapping-sample.png}
    \caption{不同类型的纹理(A-漫反射纹理,B-法线纹理,C-高度纹理)\cite{paltashev2014texture}}
    \label{fig:TextureMapping-sample}
\end{figure}

\begin{figure}[htbp]
    \centering
    \includegraphics[width=\textwidth]{figures/Nanite.png}
    \caption{通过Nanite技术渲染的超高精度模型\cite{Nanite}}
    \label{fig:Nanite}
\end{figure}

随着图形技术的发展,计算机实时渲染的视觉质量已经接近视觉特效和电影制作的水平,这些质量改进是通过采用电影渲染中使用的方法来实现的,
例如基于物理的着色技术、光线追踪技术和用于精确全局照明的去噪技术,以及如图\ref{fig:Nanite}所示,类似 Nanite 的超高精度模型渲染技术。
这些高精度渲染技术依赖的高质量纹理占用了大量空间,例如《孤岛惊魂6》的游戏本体大小为91GB,而其高清纹理扩展包大小达到62GB,约为游戏本体大小的三分之二\cite{FarCry6}。
除了巨大的磁盘空间占用外,纹理在渲染过程中占用大量显存,这导致对显存大小以及带宽的较高要求。
因此各种纹理压缩技术被广泛采用,以减轻显存和带宽的压力。
在这种情况下,纹理存储在磁盘上并以某种压缩格式传输到 GPU 上。
解压缩仅发生在 GPU 上,这种方法节省了大量内存、显存以及带宽。
纹理压缩还可以降低功耗,因为内存与显存之间传输数据带来的功耗对于笔记本电脑、平板电脑和智能手机等移动设备尤为可观。

与传统图像不同,纹理需要在 GPU 端进行采样,这个过程具有高度随机性。渲染期间仅采样纹理的所需部分,并且事先不知道顺序。
此外,三维表面上三角形的相邻并不意味着相应纹理区域的相邻(见图\ref{fig:TextureMapping})。
因此,渲染过程的整体性能高度依赖于纹理访问的效率。
由于缺乏对随机纹素访问的支持,标准压缩算法(RLE、LZW、Deflate) 和流行的压缩图像格式 (JPEG、PNG、TIFF)不适用于纹理,
这些算法如果不解压整个纹理就无法取得特定的纹素。
因此纹理压缩技术应运而生,通过高效的压缩算法减少显存占用和带宽消耗,同时尽可能保持视觉质量,从而提高渲染性能,成为计算机图形学中的关键技术之一。

随机访问决定了各种纹理压缩格式的主要特性。
大多数纹理压缩方案将整个图像划分为固定大小的块单独压缩,这类方案称为块压缩。
目前在不同平台上已经出现了多种不同的纹理块压缩格式:用于PC平台的DXT格式(包含BC1至BC7共7种格式),用于移动设备的ETC、PVRTC、以及ASTC格式。
这些格式在对应平台上支持GPU硬件解码与采样,因此具有较少的性能开销。
由于纹理的多样性,这些格式对每个纹理块都具有多种不同的编码配置,例如不同的分区压缩方案等。
这些编码配置增加了编码(压缩)过程的复杂性,但也使得对不同类型的纹理都具有较好的压缩效果。

最近已经出现了许多利用神经网络进行纹理压缩的工作,但由于解压缩过程需要进行网络推理,对于GPU要求较高,目前并未得到较为广泛的应用。

\section{问题与挑战}

对于艺术家通过建模软件导出的原始颜色纹理、法线纹理、高度纹理等,目前的块压缩方法已经可以以较高质量进行压缩。
对于神经网络模型中张量形式的神经特征,同样可以视作纹理并使用块压缩方法进行压缩,
但由于现有纹理块压缩工具例如微软的DirectTex、英伟达的GPU纹理压缩工具NVTT、Intel的ISPC纹理压缩器等都不支持
基于自动微分的可微计算,因此无法在神经网络模型的优化过程中直接加入现有的块压缩编解码器,
只能在优化完成后对神经特征进行块压缩,这会较为显著地增加模型的目标误差。
与神经网络模型类似,基于移动基分解的光照数据压缩技术\cite{silvennoinen2021moving}将光照数据压缩为
一系列的系数与基向量矩阵,HDR纹理的RGBM编码技术将RGB格式的HDR纹理编码为RGBA格式的LDR纹理,
这两种方法也可以与纹理块压缩技术结合使用,但由于无法将块压缩的编解码过程加入并进行端到端优化,
模型整体的目标误差也会显著增加。

为了使纹理块压缩技术可以与更多方法结合使用,需要一个对现有纹理块压缩技术的可微近似,即可微的纹理块压缩编解码器。
目前已有相关工作对这个问题进行了研究,BCf\cite{weinreich2024real}提出了一种可微BC6解码器,
并在神经特征的训练期间模拟 BC6 解压缩。BCf\cite{weinreich2024real}仅具有可微解码器,缺少可微编码器对编码配置的选择
因此需要在初始化时确定所有纹理块的编码配置,并在训练时固定配置。由于优化过程中神经特征不断发生变化,
固定配置对于变化的纹理不是最优解,因此这种方法没有充分利用BC6格式的编码能力。

本文提出了一种编解码架构的可微 DXT 格式纹理块压缩模型。基于 DXT 格式的定义,
设计可微的编码器、解码器,以及配置选择器,从而实现 DXT 格式编解码器的可微近似。

可微编解码器模型可以解决编码配置的选择问题,但具有两个挑战:

(1)编码过程的复杂性:DXT格式的解码过程基于线性插值,因此整体上是可微的。而编码过程不同的纹理块压缩工具具有不同实现方式,
以开源的BC7e编解码器为例,编码过程具有大量分支、循环、位运算、量化舍入等不可微过程,

(2)编码配置选择问题:标准纹理块压缩工具在编码时通常会尝试使用多种编码配置进行
编解码,并选择使得压缩误差最小的编码配置。
由于可微编解码器通常会与其他方法结合进行端到端优化,
优化过程中每次迭代都会进行一次可微编解码,因此在迭代中尝试大量编码配置的过程会带来显著的计算开销。

针对挑战(1),本文整体上简化并归纳了DXT格式编码过程的数学表达。
由于编码的量化过程不可微,本文所提出的模型忽略了量化过程的影响。

针对挑战(2),本文提出了一种基于混合专家模型的编码配置选择器。该方
法将编码配置根据亲和度通过 Top-k 排序,分为始终激活的稳定配置与通过随
机采样激活的随机配置。只需使用数量较少的稳定配置与随机配置,就能获得接近枚
举所有配置产生的编码质量,同时显著地减少计算开销。

\section{研究现状}

\subsection{纹理块压缩技术}
目前国内外对于纹理块压缩技术已有充分的探索,在成果上产生了DXT、ETC、PVRTC和ASTC等高质量的纹理块压缩格式。
DXT格式最早被称为S3TC\cite{iourcha1999system},于1999年提出,是最早的纹理块压缩方案之一,它由S3公司开发并获得专利。
微软在DirectX 6.0中将S3TC命名为DXT1。后续的DXT2-5格式增加了对Alpha通道纹理的支持。
从DirectX 10开始,这些格式被称为BC1-BC3。
BC1格式将纹理划分为4x4个块,每个块存储两个颜色端点和16个2位索引。
每个像素根据索引插值颜色端点。BC4和BC5被引入用于法线纹理压缩。
随着DirectX 11的发布,BC6H专为高动态范围(HDR)纹理而设计,BC7专为高质量压缩而设计。
在BC1的基础上,BC7支持复杂的多模式压缩设置,如块分区和旋转通道。
爱立信纹理压缩(ETC)格式专为移动设备设计,这种压缩方案的第一个版本PACKMAN\cite{strom2004packman}于2004年推出。
2005年,PACKMAN的增强版本iPACKMAN\cite{strom2005packman}发布,通常称其为ETC1。
2007年,基于ETC1改进的ETC2\cite{strom2007etc}格式被提出。
专为iOS设备设计的PowerVR纹理压缩(PVRTC)方法于2003年由Fenney\cite{fenney2003texture}提出。
自适应可缩放纹理压缩(ASTC)\cite{nystad2012adaptive}是目前最灵活的纹理压缩格式,它由ARM和AMD联合开发,于2012年推出。
ASTC与BC7类似,将每个纹理块编码为颜色端点对与插值索引,同时支持预定义的分区。
但ASTC独特的有界整数序列编码(BISE)和支持双线性上采样的权重网格等新特性使ASTC几乎成为目前最先进的纹理块压缩格式。
ASTC由于没有受到PC端主流的图形接口DirectX支持,因此尚未被广泛用于PC平台,目前主要在移动设备上使用\cite{vaidyanathan2023random}。

\subsection{神经纹理压缩}

目前已有许多与神经纹理压缩相关的研究,这类研究通常采用多分辨率特征网格作为纹理的压缩表示,并通过神经解码器将特征解码为纹素。
Instant NGP\cite{muller2022instant}使用神经网络与特征网格表示纹理等图形基元,通过可训练特征向量的多分辨率哈希编码来增强小型神经网络的能力。
Compact NGP\cite{takikawa2023compact}在Instant NGP基础上引入了索引函数,改善了性能。
NTBC\cite{fujieda2024neural}使用多分辨率的特征网格表示一种材质中的所有纹理,并采用多层感知机(MLP)将特征网格中的特征向量解码为标准DXT格式纹理,
其中解码的网络推理过程可以在渲染前进行,因此没有额外的GPU开销。
NTBC旨在减少纹理在磁盘占用的空间,即在磁盘上对传统纹理块压缩格式通过神经网络进一步压缩,但没有改善显存与带宽占用。
NTC\cite{vaidyanathan2023random}使用多分辨率的特征网格压缩表示纹理集,允许进行随机访问并按需实时解压缩,与传统纹理块压缩技术相比在相同显存占用下
获得了更好的纹理压缩质量,但需要在渲染过程中进行网络推理,这产生了额外的GPU开销。
BCf\cite{weinreich2024real}在此基础上对特征网格使用BC6格式压缩,并在训练期间模拟 BC6 解压缩,从而进一步减少了显存占用。

BCf\cite{weinreich2024real}与本文的研究密切相关,不同之处在于本文研究DXT格式中BC1至BC7这7种格式编解码器的可微近似,
而BCf\cite{weinreich2024real}仅研究BC6格式的解码过程。

\section{研究内容}

目前的块压缩方法已经可以以较高质量压缩艺术家通过建模软件导出的纹理。
但对于同样可以视作纹理的神经网络模型中张量形式的神经特征,直接使用块压缩方法压缩会造成模型整体误差
显著增加。由于现有纹理块压缩工具不支持自动微分,为了在神经网络模型的优化过程中加入块压缩的编解码过程,
从而降低纹理块压缩技术的加入导致神经网络模型增加的整体误差,本文提出了一种可微纹理块压缩模型,
通过Pytorch的C++版本实现了DXT格式编解码器的可微近似,可以整合到优化训练中进行基于梯度的端到端优化。
优化完成后再使用现有的纹理块压缩工具对神经特征进行压缩,运行时通过GPU硬件解码纹理格式的神经特征,
不会带来额外的计算开销。
本文提出的方法也适用于基于移动基分解的光照数据压缩技术\cite{silvennoinen2021moving}和
HDR纹理的RGBM编码技术。

具体来讲,本文进行了以下两个工作:

其一,本文提出了一种DXT格式的可微纹理块压缩模型,支持BC1至BC7,包含可微编解码器,
可以作为DXT格式的可微近似整合到基于梯度的优化过程中并通过自动微分实现端到端优化,
从而减少纹理块压缩的加入导致模型增加的整体误差。
本文提出的方法相较于BCf\cite{weinreich2024real}的仅解码器模型,
通过加入可微编码器支持了优化过程中编码配置的动态选择,
更充分地利用了纹理块压缩格式的压缩能力。

其二,针对编码配置选择的计算效率问题,本文提出了一种基于混合专家模型(Mixture of Experts)的编码配置选择器。
相比于遍历所有编码配置的枚举方法,在压缩质量几乎不变的情况下,这个模块使得计算效率提升了一个数量级。

最后在神经纹理压缩\cite{weinreich2024real}、基于移动基分解的光照数据压缩\cite{silvennoinen2021moving}、基于RGBM编码的HDR纹理压缩,三个应用场景中进行实验,以证明本文所提出方法的有效性。
本文提出的方法一定程度上解决了纹理块压缩技术的可微性和高效性问题,使得纹理块压缩过程可以结合到其他压缩模型的训练中,
并进行端到端的高效训练,从而扩展了纹理块压缩技术的应用场景。

\section{论文的组织结构}

本文由五个章节组成,其组织结构和主要内容如下:

第一章:绪论。本章首先介绍了本文的研究背景、问题与挑战,然后分别阐述纹理
块压缩技术和神经纹理压缩技术的研究现状,最后总结了本文的主要工作。

第二章:相关工作。本章主要介绍本研究相关的背景知识和相关工作。
首先介绍了DXT纹理格式与混合专家模型。
然后对实验部分使用的神经纹理压缩技术、基于移动基分解的光照数据压缩技术、HDR纹理的RGBM编码技术三个应用场景进行概述。

第三章:可微纹理块压缩模型。本章主要介绍本文所提出的可微纹理块压缩模型,
从标准DXT格式出发,简化并归纳了编解码算过程的数学形式,然后构建了可微纹理块压缩模型的整个框架。
最后在三个应用场景中进行了实验,验证了所提出方法的有效性。

第四章:基于混合专家模型的编码配置选择器。本章首先分析了
第三章中可微纹理块压缩模型存在的问题,即编码配置选择过程的较大计算开销。
然后引入了混合专家模型,在几乎不损失压缩质量的前提下,显著提高了计算效率。
最后在三个应用场景中对提出的模型进行了实验,验证了所提出方法的有效性。

第五章:总结与展望。本章节对本文的研究内容进行了总结,并针对不足之处
提出未来可能的改进。

