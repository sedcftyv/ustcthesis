% !TeX root = ../main.tex

\chapter{绪论}

\section{研究背景与意义}
纹理(Texture)是三维渲染过程中使用的图像数据,该图像的各个像素称为纹素。如图\ref{fig:TextureMapping}所示,最简单的纹理是映射到三维表面的二维图像。
纹理不仅可以存储颜色,还可以存储高度、法线方向、材质属性(金属度、粗糙度)以及光照等信息。
如图\ref{fig:TextureMapping-sample}所示,A、B和C分别为漫反射纹理、法线纹理与高度纹理,漫反射纹理决定物体在标准光照条件下的外观、法线纹理为其增添阴影的视觉效果、高度纹理用于模拟表面凹凸的效果。

\begin{figure}[htbp]
    \centering
    \includegraphics[width=\textwidth]{figures/TextureMapping.png}
    \caption{纹理映射示例(三角形的相邻并不意味着相应纹理区域的相邻)}
    \label{fig:TextureMapping}
    \figurenote{需要重画}
\end{figure}

\begin{figure}[htbp]
    \centering
    \includegraphics[width=\textwidth]{figures/TextureMapping-sample.png}
    \caption{不同类型的纹理(A-漫反射纹理,B-法线纹理,C-高度纹理)}
    \figurenote{需要重画}
    \label{fig:TextureMapping-sample}
\end{figure}

\begin{figure}[htbp]
    \centering
    \includegraphics[width=\textwidth]{figures/Nanite.png}
    \caption{通过Nanite技术渲染的超高精度模型}
    \figurenote{需要引用}
    \label{fig:Nanite}
\end{figure}

随着图形技术的发展,计算机实时渲染的视觉质量已经接近视觉特效和电影制作的水平,这些质量改进是通过采用电影渲染中使用的方法来实现的,
例如基于物理的着色技术、光线追踪技术和用于精确全局照明的去噪技术,以及如图\ref{fig:Nanite}所示,类似 Nanite 的超高精度模型渲染技术。
这些高精度渲染技术依赖的高质量纹理占用了大量空间,例如《孤岛惊魂6》的游戏本体大小为91GB,而其高清纹理扩展包大小达到62GB,约为游戏本体大小的三分之二\cite{FarCry6}。
除了巨大的磁盘空间占用外,纹理在渲染过程中占用大量显存,这导致对显存大小以及带宽的较高要求。
因此各种纹理压缩技术被广泛采用,以减轻显存和带宽的压力。
在这种情况下,纹理存储在磁盘上并以某种压缩格式传输到 GPU 上。
解压缩仅发生在 GPU 上,这种方法节省了大量内存、显存以及带宽。
纹理压缩还可以降低功耗,因为内存与显存之间传输数据带来的功耗对于笔记本电脑、平板电脑和智能手机等移动设备尤为可观。

与传统图像不同,纹理需要在 GPU 端进行采样,这个过程具有高度随机性。渲染期间仅采样纹理的所需部分,并且事先不知道顺序。
此外,三维表面上三角形的相邻并不意味着相应纹理区域的相邻(见图\ref{fig:TextureMapping})。
因此,渲染过程的整体性能高度依赖于纹理访问的效率。
由于缺乏对随机纹素访问的支持,标准压缩算法 (RLE、LZW、Deflate) 和流行的压缩图像格式 (JPEG、PNG、TIFF) 不适用于纹理,
这些算法如果不解压整个纹理就无法取得特定的纹素。
因此纹理压缩技术应运而生,通过高效的压缩算法减少显存占用和带宽消耗,同时尽可能保持视觉质量,从而提高渲染性能,成为计算机图形学中的关键技术之一。

随机访问决定了各种纹理压缩格式的主要特性。
大多数纹理压缩方案将整个图像划分为固定大小的块单独压缩,这类方案称为块压缩。
目前在不同平台上已经出现了多种不同的纹理块压缩格式:用于PC平台的DXT格式,用于移动设备的ETC、PVRTC、以及ASTC格式。
这些格式在对应平台上支持GPU硬件解码与采样,因此具有较少的性能开销。

最近已经出现了许多利用神经网络进行纹理压缩的工作,但由于解压缩过程需要进行网络推理,对于GPU要求较高,目前并未得到较为广泛的应用。

目前纹理类型大致可分为以下两类:(1)艺术家通过建模软件导出的原始颜色纹理、法线纹理、高度纹理等。
(2)通过某种压缩方法得到的纹理形式的中间数据,例如对(1)中的纹理集通过神经网络压缩得到的特征网格、
对光照数据使用可微基分解技术压缩得到的系数与基向量矩阵、
对HDR纹理使用RGBM编码得到的编码数据等。

对于第(1)类纹理目前的块压缩方法已经可以以较高质量进行压缩。
但对于第(2)类纹理直接使用现有的块压缩方法会导致较大的质量损失,
原因是这一类纹理由于本身作为一种压缩表示,在GPU采样后还需要进行额外的一次解码过程(例如从特征网格采样的特征向量需要进行网络推理、基分解得到的系数与基向量采样后需要进行插值、RGBM编码数据采样后需要进行解码),
而目前的纹理块压缩工具的压缩过程无法考虑到这个额外解码过程对压缩误差的影响。例如将经过神经网络训练得到的特征网格使用纹理块压缩技术压缩时,块压缩方法无法考虑到特征矩阵被压缩对网络推理的影响。

其中一个较好的解决方法是将纹理块压缩技术与生成第(2)类纹理的过程结合,并进行端到端优化,使得第(2)类纹理在生成过程就能考虑到纹理块压缩带来的影响。
因此需要一个对现有纹理块压缩技术的可微近似,即可微的纹理块压缩工具。
由于目前的纹理块压缩工具例如微软的DirectTex、英伟达的GPU纹理压缩工具NVTT、Intel的ISPC纹理压缩器等都不支持自动微分,
本文针对PC平台的DXT格式,研究其较好的可微近似。


\section{研究现状}

\subsection{纹理块压缩技术}
目前国内外对于纹理块压缩技术已有充分的探索,在成果上产生了DXT、ETC、PVRTC和ASTC等高质量的纹理块压缩格式。
DXT格式最早被称为S3TC\cite{iourcha1999system},于1999年提出,是最早的纹理块压缩方案之一,它由S3公司开发并获得专利。
微软在DirectX 6.0中将S3TC命名为DXT1。后续的DXT2-5格式增加了对Alpha通道纹理的支持。
从DirectX 10开始,这些格式被称为BC1-BC3。
BC1格式将纹理划分为4x4个块,每个块存储两个颜色端点和16个2位索引。
每个像素根据索引插值颜色端点。BC4和BC5被引入用于法线纹理压缩。
随着DirectX 11的发布,BC6H专为高动态范围(HDR)纹理而设计,BC7专为高质量压缩而设计。
在BC1的基础上,BC7支持复杂的多模式压缩设置,如块分区和旋转通道。
爱立信纹理压缩(ETC)格式专为移动设备设计,这种压缩方案的第一个版本PACKMAN\cite{strom2004packman}于2004年推出。
2005年,PACKMAN的增强版本iPACKMAN\cite{strom2005packman}发布,通常称其为ETC1。
2007年,基于ETC1改进的ETC2\cite{strom2007etc}格式被提出。
专为iOS设备设计的PowerVR纹理压缩(PVRTC)方法于2003年由Fenney\cite{fenney2003texture}提出。
自适应可缩放纹理压缩(ASTC)\cite{nystad2012adaptive}是目前最灵活的纹理压缩格式,它由ARM和AMD联合开发,于2012年推出。
ASTC与BC7类似,将每个纹理块编码为颜色端点对与插值索引,同时支持预定义的分区。
但ASTC独特的有界整数序列编码(BISE)和支持双线性上采样的权重网格等新特性使ASTC几乎成为目前最先进的纹理块压缩格式。
ASTC由于没有受到PC端主流的图形接口DirectX支持,因此尚未被广泛用于PC平台,目前主要在移动设备上使用\cite{vaidyanathan2023random}。

\subsection{神经纹理压缩}

目前已有许多与神经纹理压缩相关的研究,这类研究通常采用多分辨率特征网格作为纹理的压缩表示,并通过神经解码器将特征解码为纹素。
Instant NGP\cite{muller2022instant}使用神经网络与特征网格表示纹理等图形基元,通过可训练特征向量的多分辨率哈希编码来增强小型神经网络的能力。
Compact NGP\cite{takikawa2023compact}在Instant NGP基础上引入了索引函数,改善了性能。
NTBC\cite{fujieda2024neural}使用多分辨率的特征网格表示一种材质中的所有纹理,并采用多层感知机(MLP)将特征网格中的特征向量解码为标准DXT格式纹理,
其中解码的网络推理过程可以在渲染前进行,因此没有额外的GPU开销。
NTBC旨在减少纹理在磁盘占用的空间,即在磁盘上对传统纹理块压缩格式通过神经网络进一步压缩,但没有改善显存与带宽占用。
Vaidyanathan等\cite{vaidyanathan2023random}使用多分辨率的特征网格压缩表示纹理集,允许进行随机访问并按需实时解压缩,与传统纹理块压缩技术相比在相同显存占用下
获得了更好的纹理压缩质量,但需要在渲染过程中进行网络推理,这产生了额外的GPU开销。
Weinreich等\cite{weinreich2024real}在此基础上对特征网格使用BC6格式压缩,并在训练期间模拟 BC6 解压缩,从而进一步减少了显存占用。

Weinreich等的研究与本文的研究密切相关,不同之处在于本文研究DXT格式中BC1至BC7这7种格式编解码器的可微近似,而Weinreich等的研究仅包括BC6格式解码器的可微近似。

\section{研究内容}

标准的纹理块压缩编解码器主要关注降低纹理本身的压缩误差。
然而对于一些纹理形式的中间数据,由于其本身已经是某种压缩表达,我们关注的并非这类纹理本身的压缩误差,而是其经过解码与原始数据产生的整体压缩误差。
以神经纹理压缩为例,在神经网络训练完成后,若将特征网格作为纹理进行块压缩,GPU采样后需要对特征向量进行网络推理从而完成解码,
此时我们关注的并非特征网格本身的压缩误差,而是特征向量经过网络推理的解码结果与原始纹理的压缩误差。

为了解决标准纹理块压缩编解码器不适合压缩这类中间数据的问题,已经出现了可微BC6解码器的相关工作\cite{weinreich2024real},
而本文的研究支持所有的DXT格式,包括BC1至BC7这7种格式的可微编解码过程。

具体来讲,本文进行了以下两个工作:

其一,本文提出了一种DXT格式的可微纹理块压缩框架,支持BC1至BC7,包含可微编解码器,
可以作为DXT格式的可微近似整合到优化训练中进行基于梯度的端到端联合优化,从而减少纹理块压缩带来的压缩误差。
该框架只用于训练过程,渲染时直接使用GPU硬件纹理解码,不会带来额外的运行时计算开销。

其二,针对最佳压缩编码配置选择的计算效率问题,本文提出了一种基于混合专家模型(Mixture of Experts)的编码配置选择器。
相比于暴力方法,在压缩质量几乎不变的情况下,这个模块使得计算效率提升了一个数量级。

最后在神经纹理压缩、基于移动基分解的光照数据压缩、基于RGBM编码的HDR纹理压缩,三个应用场景中进行实验,以证明本文所提出方法的有效性。
本文解决了纹理块压缩编码的可微性和高效性问题,使得纹理块压缩过程可以结合到其他压缩模型的训练中,
通过端到端的高效训练更好地挖掘出纹理块压缩技术的优化空间。

\section{论文的组织结构}

本文由五个章节组成,其组织结构和主要内容如下:
第一章:绪论。本章首先介绍了本文的研究背景与意义,然后分别阐述纹理
块压缩技术和神经纹理压缩的研究现状,最后总结了本文的主要工作。

第二章:相关工作。本章主要介绍本研究相关的背景知识和相关工作。
首先介绍了DXT纹理格式与混合专家模型。
然后对实验部分使用的神经纹理压缩、基于移动基分解的光照数据压缩、基于RGBM编码的HDR纹理压缩三个应用场景进行概述。

第三章:可微纹理块压缩模型。本章主要介绍本文所提出的可微纹理块压缩模型,
从标准DXT格式出发,推导了编解码计算过程的连续形式,然后构建了可微纹理块压缩模型的整个框架。
最后在三个应用场景中进行了实验,验证了所提出方法的有效性。

第四章:基于混合专家模型的编码配置选择器。本章首先分析了
上一章中可微纹理块压缩模型存在的问题,即多个编码配置选择过程的较大计算开销。
然后本章节引入了混合专家模型,在几乎不损失压缩质量的前提下,很大程度提高了计算效率。
最后本章节同样在三个应用场景中对提出的模型进行了实验,验证了所提出方法的有效性。

第五章:总结与展望。本章节对本文的研究内容进行了总结,并针对不足之处
提出未来可能的改进。

