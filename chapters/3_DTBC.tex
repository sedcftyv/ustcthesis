% !TeX root = ../main.tex

\chapter{可微纹理块压缩模型}

\section{引言}

为了进一步节省显存,在神经网络优化结束后,可以对神经特征使用纹理块压缩技术进行压缩。
但由于现有纹理块压缩工具不支持自动微分,无法与神经网络这类优化方法联合训练,
因此训练过程无法感知到块压缩带来的影响,最终块压缩会导致整体模型的目标误差显著增加。
其中一个解决办法是根据压缩格式定义,实现可微的编解码器,并将其与神经网络模型
结合,进行端到端联合优化,从而降低块压缩对网络模型造成的额外质量损失。
目前已有基于可微BC6解码器的应用研究,但由于仅解码器框架的局限性,
只能进行固定配置优化。本章提出了一种新颖的可微纹理块压缩模型,包括DXT格式的编码器与解码器,
可以根据编码误差来选择最佳编码配置。

\section{问题描述}
\label{问题描述}

本章提出的模型旨在实现可微的DXT格式编解码器。
将压缩前的原始纹理定义为$\mathbf{T}'\in \mathbb{R}^{w\times h\times c}$,
压缩后的纹理定义为$\mathbf{T}\in \mathbb{R}^{w\times h\times c}$,
其中$w$、$h$、$c$分别表示纹理的宽度、高度、通道数。
DXT格式编解码器的框架整体上可以使用如下形式表示:
\begin{equation}
    \mathbf{T}=\text{ConfigSelector}(\text{Decoder}(\text{Encoder}(\mathbf{T}')))
\end{equation}
其中Encoder表示可微编码器,Decoder表示可微解码器,ConfigSelector表示编码配置选择器。可微编码器首先将纹理分为4x4大小的纹理块,
然后对于每个纹理块,使用不同的编码配置进行编码,每个编码配置经过可微解码器解码得到对应的纹理,
最后编码配置选择器选择编码误差最小的结果作为输出。
可微的编解码器使用时会结合到其他模型中进行端到端联合优化,优化完成后,
对$\mathbf{T}'\in \mathbb{R}^{w\times h\times c}$使用标准DXT格式编码器即可得到较好的压缩格式数据。
由于量化过程不可微,因此本章提出的模型忽略了量化过程。

DXT格式共包含BC1至BC7共7种格式,这些格式都将纹理划分为
多个4x4的块进行压缩存储。由于DXT格式的共同特点是压缩时存储颜色端点与权重,解压时根据权重对颜色端点进行线性插值。
因此在编码过程中,对于BC1至BC7这7种格式,在不考虑量化的情况下,需要解决的共同问题 $P_{m,n}$ 是:给定输入纹素组 $\mathbf{C}\in\mathbb{R}^{m\times n}$,
找到适合表达该组纹素的线段端点 $\mathbf{e}_0\in\mathbb{R}^n$ 和 $\mathbf{e}_1\in\mathbb{R}^n$ 
以及插值系数 $\mathbf{t}\in\mathbb{R}^m$。其中 $m$ 表示纹素个数,$n$ 表示纹素通道数。

对于 BC1 和 BC4,编码问题分别为 $P_{16,3}$ 和 $P_{16,1}$。
BC3 的编码问题可以拆分为 $P_{16,3}$ 和 $P_{16,1}$。
而 BC2 可以看成是 BC3 的特化形式,
即 alpha 通道的线段端点是 $0$ 和 $1$。
BC5 相当于两个通道分别使用BC4格式。BC6 包含单分区($P_{16,3}$)和双分区配置,
而双分区配置编码问题可以拆分为 $P_{m_1,3}$ 和 $P_{m_2,3}$,
其中 $m_1$ 和 $m_2$ 是两个分区的纹素个数。

BC7 包含模式0至模式7共8种编码模式,其中模式0到模式3用于压缩 3 通道纹理,
而模式4到模式7用于压缩 4 通道纹理。模式1和模式3使用了双分区配置,
其编码问题都同于BC6的双分区配置;模式0和模式2是三分区配置,
编码问题可以拆分为 $P_{m_1,3}$、$P_{m_2,3}$ 和 $P_{m_3,3}$,
其中 $m_1$、$m_2$ 和 $m_3$ 是三个分区的纹素个数;模式4和模式5包含 4 种旋转配置,
每个旋转配置的编码问题同于 BC3;模式6的编码问题为 $P_{16,4}$;
模式7是双分区配置,其编码问题可以拆分为  $P_{m_1,4}$ 和 $P_{m_2,4}$。

综上所述,BC7中用于压缩RGBA纹理,包含了分区和旋转的模式4至模式7可以看作DXT格式中最复杂的格式。
不考虑量化的情况下,其他编码格式相当于BC7中模式4至模式7的子集,
可微BC7格式的RGBA编解码器可以较为容易地推广到其他DXT格式中。
因此本文主要关注BC7格式的RGBA编解码器。专为HDR纹理设计的BC6格式
十分适合压缩浮点数形式的神经特征,本文也实现了BC6格式的可微编解码器。


\section{模型框架}

\subsection{整体框架}

整体框架如图\ref{fig:DBC_overview}所示,
图中展示了可微编解码器架构(DBC),并以神经纹理压缩为例,
展示了可微编解码器与其他压缩方法结合的管线。
其中最左边的输入纹理为神经纹理压缩中的纹理形式的特征网格,
经过可微编码器后通过不同编码配置进行了编码,之后每种编码经过
可微解码器进行解码,然后使用配置选择器根据编码误差选择最佳编码配置,
并输出对应的编解码结果,之后使用神经网络对可微编解码后的特征网格
进行解码,得到最终的解码纹理。

需要注意的是,由于本文所提出的可微编解码器本身没有需要优化的参数,
因此结合到其他压缩方法并进行联合优化时,损失函数仅为整体误差。

\begin{figure}[htbp]
    \centering
    \includegraphics[width=1\textwidth]{figures/DBC_overview_v2.pdf}
    \caption{整体框架}
    \label{fig:DBC_overview}
\end{figure}


\subsection{可微编解码器}

在可微编码中,\ref{问题描述}中的公共问题 $P_{m,n}$ 可以通过主成分分析(PCA)来解决。
首先,计算中心化的纹素组 $\mathbf{C}_{\text{center}}\in\mathbb{R}^{m\times n}$:

\begin{equation}
\mathbf{C}_{\text{center}}=\mathbf{C}-\text{mean}(\mathbf{C})
\end{equation}
然后,对 $\mathbf{C}_{\text{center}}$ 进行奇异值分解(SVD):
\begin{equation}
\mathbf{C}_{\text{center}}=\mathbf{U}\text{diag}(\mathbf{s})\mathbf{V}^T
\end{equation}
其中,$\mathbf{U}\in \mathbb{R}^{m \times m}$,$\mathbf{s}\in \mathbb{R}^{n}$,$\mathbf{V}\in \mathbb{R}^{n\times n}$,$\text{diag}(\mathbf{s})\in\mathbb{R}^{n\times n}$ 表示以 $\mathbf{s}$ 为对角元素的对角矩阵。
选择 $\mathbf{V}$ 中最大奇异值对应的奇异向量 $\mathbf{v}'\in \mathbb{R}^{n}$ 作为主方向。
接着,将中心化的纹素组 $\mathbf{C}_\text{center}$ 投影到主方向 $\mathbf{v}'$ 上,
得到权重 $\mathbf{t}'\in\mathbb{R}^{m}$:
\begin{equation}
\mathbf{t}'=\mathbf{C}_{\text{center}}\cdot\mathbf{v}'
\end{equation}
从而可以得到两个颜色端点 $\mathbf{e}'_0\in\mathbb{R}^n$ 和 $\mathbf{e}'_1\in\mathbb{R}^n$:
\begin{equation}
\mathbf{e}'_0=\text{mean}(\mathbf{C})+\text{min}(\mathbf{t}')\mathbf{v}\\
\mathbf{e}'_1=\text{mean}(\mathbf{C})+\text{max}(\mathbf{t}')\mathbf{v}
\end{equation}
由于端点的数值范围应在 $[0,1]$ 之间,因此需要对端点进行 clamp 操作。
为了保证训练时梯度在端点超出有效数值范围时非零,使用 $\text{leakyclamp}$,从而得到近似处于有效范围的端点 $\mathbf{e}_0$ 和 $\mathbf{e}_1$:
\begin{equation}
\mathbf{e}_0=\text{leakyclamp}(\mathbf{e}'_0,0,1)\\
\mathbf{e}_1=\text{leakyclamp}(\mathbf{e}'_1,0,1)
\end{equation}
\begin{equation}
\text{leakyclamp}(x,a,b)=\text{clamp}(x,a,b)\\
                        +\epsilon[(\max(x,b)-b)+(\min(x,a)-a)]
\end{equation}
其中$\epsilon=0.01$。

然后需要重新计算主方向 $\mathbf{v}\in\mathbb{R}^n$ 并重新投影得到权重 $\mathbf{t}''\in\mathbb{R}^m$:
\begin{equation}
\mathbf{v}=\frac{\mathbf{e}_1-\mathbf{e}_0}{\|\mathbf{e}_1-\mathbf{e}_0\|}\\
\mathbf{t}''=\frac{(\mathbf{C}-\mathbf{e}_0)\cdot\mathbf{v}}{\|\mathbf{e}_1-\mathbf{e}_0\|}
\end{equation}
同样需要用 $\text{leakyclamp}$ 使权重 $\mathbf{t}''$ 处于有效数值范围 $[0,1]$ 内,即
\begin{equation}
\mathbf{t}=\text{leakyclamp}(\mathbf{t}'',0,1)
\end{equation}
解码器进行插值得到解码的纹素组 $\mathbf{C}'\in\mathbb{R}^{m\times n}$ 为
\begin{equation}
\mathbf{C}'=\mathbf{e}_0(1-\mathbf{t})+\mathbf{e}_1\mathbf{t}
\end{equation}

解决了公共问题$P_{m,n}$后,每种DXT格式的编解码过程可以看作一系列
$P_{m,n}$的组合,例如BC7中模式7具有的64种双分区对应64个$P_{m_1,n}$和
$P_{m_2,n}$问题。

接下来需要解决编码配置选择的问题,例如对于BC7中模式7具有的64种双分区,
需要从中选择压缩误差最小的一种分区方案。

\subsection{配置选择器}

设输入纹理块 $\mathbf{B}\in\mathbb{R}^{b\times n}$,其中 $b$ 是一个纹理块中的纹素个数。
所有配置的集合为 $\mathcal{I}$,针对配置 $i\in\mathcal{I}$ 经过可微编解码后得到的压缩纹理块为 $\mathbf{B}'_i\in\mathbb{R}^{b\times n}$。
定义编码配置$i$对应的压缩误差 $L_i$ 为
\begin{equation}
L_i=\|\mathbf{B}-\mathbf{B}'_i\|_F^2
\end{equation}

配置选择器的功能是对每个4x4纹理块$\mathbf{B}$,在不同编码配置中选择最佳配置,
即应该选择使得压缩误差$L_i$最小的编码配置$i^*$:
\begin{equation}
    i^*=\mathop{\arg\min}\limits_{i} L_i
\end{equation}
由于在计算压缩误差之前,已经对每种编码配置都进行了可微编解码,配置选择器可以
直接将使用了$i^*$进行编解码得到的$\mathbf{B}'_{i^*}$作为输出。
\begin{equation}
    \mathbf{B}'_{i^*}=\mathop{\arg\min}\limits_{\mathbf{B}'_i} L_i
\end{equation}
由于$\mathop{\arg\min}$是不可微的,因此使用加权混合代替$\mathop{\arg\min}$:
\begin{equation}
\mathbf{B}'_{i^*}=\sum_{i\in\mathcal{I}} w_i\mathbf{B}'_i
\end{equation}
其中 $w_i$ 是配置 $i$ 的组合系数。由于最终只选择使得$L_i$最小的配置,
因此实际上 $w_i$ 是 one-hot 的:
\begin{equation}
    w_i =\left\{\begin{matrix}
        1,& i=\mathop{\arg\min}\limits_{i} L_i
        \\0,& i\ne\mathop{\arg\min}\limits_{i} L_i
        \end{matrix}\right.
\end{equation}
这种方式的缺点是为了确定最佳的编码配置,在可微编解码过程中需要使用所有的编码配置进行编解码,
才能计算出每个编码配置对应的压缩误差$L_i$,这产生了比较显著的计算开销,下一章中将针对这一问题进行改进。



\section{实验设计与分析}

本文使用Pytorch的C++版本实现了可微BC7 RGBA模式的编解码器以及可微BC6编解码器,
并在神经纹理压缩、基于移动基分解的光照数据压缩、HDR纹理的RGBM编码
三个应用场景上进行实验。
其中神经纹理压缩实验使用可微BC6编解码器,其余两个实验使用可微BC7 RGBA模式编解码器。
可微BC6与BC7编解码器都通过Pytorch实现,以支持自动微分。

\subsection{神经纹理压缩实验}

\subsubsection{实验设置}

\paragraph{实验数据集}

在神经纹理模型的实验中,使用了来自polyhaven.com\cite{PolyHaven}的 12 个公开的材质纹理集,
每个纹理集包含 8 个通道,由 3 张纹理组成:3通道的漫反射问题、2通道的法线纹理 和 3通道的ARM纹理(环境光遮蔽、粗糙度、金属度),
分辨率均为2048×2048(2K)。

\paragraph{评估指标}

纹理压缩任务常用的评估指标是峰值信噪比(Peak Signal-to-Noise Ratio, PSNR)、
结构相似性(Structural SIMilarity, SSIM)\cite{wang2004image},
以及FLIP\cite{andersson2020flip}。

峰值信噪比(Peak Signal-to-Noise Ratio, PSNR)是图像和视频处理领域一种常用的客观质量评价指标。
它用于衡量经过压缩或重建后的图像与原始图像之间的差异,PSNR 值越高,
说明失真越小,图像质量越好。

PSNR 的计算基于均方误差(Mean Squared Error, MSE),其定义如下:
\begin{equation}
\text{MSE} = \frac{1}{MN} \sum_{i=0}^{M-1} \sum_{j=0}^{N-1} \left[ I(i,j) - K(i,j) \right]^2
\end{equation}
其中,$M$ 和 $N$ 分别为图像的宽度和高度,$I(i,j)$ 和 $K(i,j)$ 分别表示原始图像和经过处理后的图像在像素 $(i,j)$ 处的灰度值。

基于 MSE,PSNR 的计算公式如下:
\begin{equation}
    \text{PSNR} = 10 \cdot \log_{10} \left( \frac{MAX^2}{\text{MSE}} \right)
\end{equation}
其中,$MAX$ 表示像素的最大可能值,例如 8-bit 图像中 $MAX = 255$。

一般而言,PSNR 数值范围通常在 30–50 dB 之间,具体含义如下:
\begin{itemize}
    \item PSNR $\geq 40$ dB:高质量,几乎无损,人眼难以察觉图像差异;
    \item 30 dB $\leq$ PSNR $<$ 40 dB:质量良好,可能存在轻微失真,但视觉上仍可接受;
    \item PSNR $<$ 30$ dB$:图像质量较差,失真较明显。
\end{itemize}

虽然 PSNR 计算简单且易于理解,但它仅基于像素误差衡量图像质量,而未考虑人眼视觉系统(HVS)的感知特性。例如,两幅图像的 PSNR 可能相同,但主观视觉效果可能存在明显差异。

结构相似性指数(Structural Similarity Index, SSIM)是一种衡量图像质量的客观指标,旨在模拟人类视觉系统(Human Visual System, HVS)对图像质量的感知。与峰值信噪比(PSNR)不同,SSIM 不仅关注像素间的数值差异,还考虑了图像的结构信息,从而提供更符合人眼感知的质量评估。

SSIM 通过亮度(Luminance)、对比度(Contrast)和结构(Structure)三个分量计算图像相似度,其数学定义如下:
\begin{equation}
\text{SSIM}(x, y) = [l(x, y)]^\alpha \cdot [c(x, y)]^\beta \cdot [s(x, y)]^\gamma
\end{equation}
其中:

$l(x, y)$ 为亮度比较函数:
\begin{equation}
l(x, y) = \frac{2\mu_x\mu_y + C_1}{\mu_x^2 + \mu_y^2 + C_1}
\end{equation}
其中 $\mu_x$ 和 $\mu_y$ 分别为图像 $x$ 和 $y$ 的均值,$C_1$ 是稳定常数。

$c(x, y)$ 为对比度比较函数:
\begin{equation}
c(x, y) = \frac{2\sigma_x\sigma_y + C_2}{\sigma_x^2 + \sigma_y^2 + C_2}
\end{equation}
其中 $\sigma_x$ 和 $\sigma_y$ 分别为图像 $x$ 和 $y$ 的标准差,$C_2$ 是稳定常数。

$s(x, y)$ 为结构相似性函数:
\begin{equation}
s(x, y) = \frac{\sigma_{xy} + C_3}{\sigma_x \sigma_y + C_3}
\end{equation}
其中 $\sigma_{xy}$ 为图像 $x$ 和 $y$ 之间的协方差,$C_3$ 是稳定常数。


通常,SSIM 取 $\alpha = \beta = \gamma = 1$,则其简化形式如下:
\begin{equation}
    \text{SSIM}(x, y) = \frac{(2\mu_x\mu_y + C_1)(2\sigma_{xy} + C_2)}{(\mu_x^2 + \mu_y^2 + C_1)(\sigma_x^2 + \sigma_y^2 + C_2)}
\end{equation}

SSIM 的取值范围为 $[-1,1]$,其中:
\begin{itemize}
    \item SSIM = 1 表示两幅图像完全相同;
    \item SSIM 接近 1 表示两幅图像高度相似;
    \item SSIM 接近 0 表示两幅图像存在较大差异;
    \item SSIM 低于 0 可能表示强烈的结构失真。
\end{itemize}

相比于 PSNR,SSIM 更符合人类视觉系统的感知特点,能够更准确地反映图像质量。
然而,SSIM 仍然存在一定局限性,例如对局部亮度变化较敏感。

FLIP是一种由 NVIDIA 提出的基于人类感知差异的图像质量评价指标,专用于计算机图形学中的渲染质量评估。
FLIP计算过程结合了人眼视觉特性和局部特征检测,以评估图像的感知质量差异。
首先,FLIP 采用对比敏感度函数(Contrast Sensitivity Function, CSF)对输入图像进行空间滤波,
以模拟人眼对不同空间频率的敏感性,并在感知均匀色彩空间中计算颜色误差,
以确保色彩差异符合人眼视觉特性。随后,FLIP 通过边缘检测和点状特征检测增强对小尺度结构(如锐利边缘、噪点等)的感知,
并采用局部对比度分析确保对高频细节的评估更符合视觉系统。
最终,FLIP 计算颜色误差 $\Delta E_c$ 和特征误差 $\Delta E_f$,并通过非线性组合:
\begin{equation}
    \Delta E = (\Delta E_c)^{1 - \Delta E_f}
\end{equation}
获得最终的感知误差值。该计算方式确保了特征误差对颜色误差的影响,
增强了对关键视觉区域(如边缘、纹理和亮度变化)的质量评估。

FLIP 的取值范围为 $[0,1]$,其中:
\begin{itemize}
    \item FLIP 值接近 0:图像几乎无损,人眼难以察觉差异。
    \item FLIP 值较高:表明图像存在可感知的质量损失,可能影响视觉体验。
    \item 局部高值区域:对应于人眼敏感的细节,如边缘、噪点、光照变化等。
\end{itemize}

\paragraph{基准模型}

为了验证本章模型的有效性,不使用可微编解码器(BC)、固定配置的可微BC6编解码器(Fix\_DTBC)
作为两个基准模型,与使用带配置选择器的可微BC6编解码器(DTBC)进行对比。
其中固定配置的可微编解码器类似于BCf\cite{weinreich2024real}的仅解码器模型。

\paragraph{实现细节}

神经网络使用与BCf\cite{weinreich2024real}中的BCf-0.5K相同的配置,
即4个3通道的特征网格,每个特征网格大小分别为512x512、256x256、128x128、64x64。
神经网络使用两层全连接结构,隐藏层通道数为16,输入通道数为12,输出通道数为8(纹理集中一组纹理的通道数之和)。
损失函数为神经网络重建的纹理与真实纹理之间的均方误差。
进行实验前首先将特征网格初始化为-1到1之间的均匀分布,神经网络权重和偏置项使用默认初始化。
将每个材质的纹理集中的一组纹理拼接成8通道的张量作为真实数据,
将2048x2048大小的整数网格作为采样的UV坐标,
训练和推理时一个epoch内,根据2048x2048大小的整数网格采样特征网格,并通过神经网络解码出原始的2K纹理。

将学习率设为0.01,使用Adam优化器先进行10000个epoch的预训练,训练过程不使用任何压缩方法。
预训练完成后,保存特征网格与网络权重参数,对于两个基准模型与本文所提出的模型分别进行实验。

对于不使用可微编解码器(BC),对特征网格直接使用标准BC6编码器进行编解码,然后进行网络推理解码原始纹理并计算误差作为结果。

对于固定配置的可微BC6编解码器(Fix\_DTBC),保持学习率为0.01不变,首先对特征网格运行一次可微BC6编解码器确定每个块的编码配置,
然后固定编码配置,将可微BC6编解码器加入到神经纹理压缩模型中,进行联合优化。

对于带配置选择器的可微BC6编解码器(DTBC),保持学习率为0.01不变,直接将可微BC6编解码器加入到神经纹理压缩模型中,进行联合优化。
加入了可微BC6编解码器的优化过程中,学习率根据损失调控,具体地,当连续 40 个 epoch 没有新的历史最低损失产生时,将学习率乘上衰减系数 0.9,每隔 50 个 epoch 将纹理使用标准BC6编解码器进行一次编解码,并计算真实损失,防止发生过拟合。当连续 5 次(对应 250 个 epoch)没有新的历史最低真实损失产生时,停止训练。
优化完成后记录历史最低的真实损失作为实验结果。

\subsubsection{实验结果与分析}

12个材质的纹理集上进行的实验结果如图\ref{fig:nm_BC6}所示,对于每个材质,
不使用可微BC6编解码器(BC)都取得了最差的结果,
而固定配置的可微BC6编解码器(Fix\_DTBC)取得了更好的结果,
带配置选择器的可微BC6编解码器(DTBC)取得了最好的结果。
加入可微BC6编解码器并进行联合优化有效降低了BC6对神经纹理压缩带来的质量损失,
同时带配置选择器的实验结果最佳,证明了编解码器框架相对与仅解码架构的优势。


\begin{figure}[htbp]
    \centering
    \includegraphics[width=1\textwidth]{figures/nm_BC6.png}
    \caption{12个材质的纹理集上进行的实验结果}
    \label{fig:nm_BC6}
\end{figure}


\subsection{基于移动基分解的光照数据压缩实验}

\subsubsection{实验设置}

\paragraph{实验数据集}

在基于移动基分解的光照数据压缩实验中,在康奈尔盒场景与4个大型开放场景中进行了测试。
Cornell Box 场景中进行了二阶球谐表示的辐照度体积实验,4个大型开放场景中进行了一阶球谐表示的辐照度体积实验。

\paragraph{评估指标}

使用结构相似度(SSIM)与FLIP作为评价指标。

\paragraph{基准模型}

为了验证本章模型的有效性,不使用可微编解码器(BC)、固定配置的可微BC7编解码器(Fix\_DTBC)
作为两个基准模型,与使用带配置选择器的可微BC7编解码器(DTBC)进行对比。

\paragraph{实现细节}

实验前首先对场景里的球谐函数进行PCA,从而初始化系数与基向量。

对于不使用可微编解码器(BC),对特征网格直接使用标准BC7编码器进行编解码,
然后进行空间滤波解码原始球谐函数并计算误差作为结果。

对于固定配置的可微BC7编解码器(Fix\_DTBC),初始化学习率为0.1,
首先对特征网格运行一次可微BC7编解码器确定每个块的编码配置,
然后固定编码配置,将可微BC7编解码器加入到MBD模型中,进行联合优化。

对于带配置选择器的可微BC7编解码器(DTBC),初始化学习率为0.1,
直接将可微BC7编解码器加入到MBD模型中,进行联合优化。

在可微BC7编解码器的优化过程中,学习率根据损失调控,具体地,
当连续 40 个 epoch 没有新的历史最低损失产生时,将学习率乘上衰减系数 0.9,
每隔 50 个 epoch 将纹理使用标准BC6编解码器进行一次编解码,并计算真实损失,
防止发生过拟合。当连续 5 次(对应 250 个 epoch)没有新的历史最低真实损失产生时,停止训练。
优化完成后记录历史最低的真实损失作为实验结果。

\subsubsection{实验结果与分析}

实验结果如图\ref{fig:MBD_BC7}所示,带配置选择器的可微BC7编解码器(DTBC)
在所有场景中相比不使用可微编解码器(BC)都取得了更好的结果,证明了
可微编解码器的有效性。

\begin{figure}[htbp]
    \centering
    \includegraphics[width=0.7\textwidth]{figures/MBD_BC7.png}
    \caption{MBD模型的实验结果(缺固定配置实验Fix\_DTBC)}
    \label{fig:MBD_BC7}
\end{figure}

\subsection{基于RGBM编码的HDR纹理压缩实验}

\section{本章小结}

本章提出了可微DXT格式编解码器模型。在引言部分,提出了将纹理块压缩方法
直接应用到其他压缩方法上面临的较高质量损失问题,阐述了可微纹理块压缩技术的
实用性。然后分析了BC1至BC7这7种格式的异同点,分析了将BC7作为DXT格式中最复杂格式
的原因。接下来提出了模型的整体框架,阐述了可微编解码器与编码配置选择器的数学表达。
之后本章在3个应用场景中对本文所提出的方法进行测试,并给出了基线比较实验,
以证明所提出的方法的有效性。


