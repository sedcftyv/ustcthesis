% !TeX root = ../main.tex

\chapter{可微纹理块压缩模型}

\section{引言}

现有的纹理块压缩方法已经可以对常见纹理进行高质量压缩。
但对于其他压缩方法产生的纹理形式的中间数据,由于现有方法无法关注到
其他压缩方法带来的整体误差,因此直接使用现有方法压缩会导致较低的压缩质量。
其中一个解决思路是根据压缩格式定义,实现可微的编解码器,并将其与其他压缩方法
结合,进行端到端联合优化,从而提高二次压缩的压缩质量。
目前已有基于可微BC6解码器的应用研究,但由于仅解码器框架的局限性,
只能进行固定配置优化。本章提出了一种新颖的可微纹理块压缩模型,包括DXT格式的编码器与解码器,
可以根据编码误差来选择最佳编码配置。

\section{问题描述}
本章提出的模型旨在实现可微的DXT格式编解码器。
首先将压缩前的原始纹理定义为$\mathbf{T}'\in \mathbb{R}^{w\times h\times c}$,
其中$w$、$h$、$c$分别表示纹理的宽度、高度、通道数。
DXT格式编解码器的框架整体上可以使用如下形式表示:

 ,并提供给上游模块。因此训练时的整体模型为
\begin{equation}
\mathbf{T}^*=\text{M}(\mathbf{T}',\theta)=\text{M}(\text{DBC}(\mathbf{T}),\theta)
\end{equation}
DBC 框架的输入是一批 4x4 的纹理块,接着我们计算各种编码配置的压缩纹理块。压缩配置可以划分成非分区的配置(Mode4-Mode6)和分区配置(Mode7)。Mode7 总共有 64 种分区配置,如果每种分区模式都计算,将带来显著的计算开销。为了解决这个问题,我们提出了 MoP 模块,其在每次 epoch 中仅计算少数几个分区配置。在压缩配置选择器中,我们根据各编码配置的压缩纹理块与输入纹理块之间的误差,随机选出压缩纹理块,并传递给上游模块。

在训练过程中,压缩后的参数纹理实际上仍以浮点形式存在。在端到端训练完成后,我们将这些压缩后的神经特征转换为 BC7 格式,使用标准的 BC7 编解码器 bc7e 进行输出。
\section{模型框架}
\subsection{可微编码器}
\subsection{可微解码器}
\section{实验设计与分析}


