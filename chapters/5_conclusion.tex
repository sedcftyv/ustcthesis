% !TeX root = ../main.tex

\chapter{总结和展望}

\section{工作总结}

本文提出了一种新颖的可微纹理块压缩模型,实现了DXT格式的可微编解码器,
可以作为DXT格式的可微近似整合到基于梯度的优化过程中并通过自动微分实现端到端优化,
从而降低纹理块压缩技术引起的模型精度损失。
同时为了减少编码配置选择过程带来的显著计算开销,提出了基于混合专家模型
的编码配置选择器,可在几乎不损失编码质量的前提下,将编码速度提升一个数量级。

具体来讲,本文首先在第二章简要介绍了DXT纹理格式与混合专家模型的基本概念与相关工作,
以及神经压缩、基于移动基分解的光照数据压缩、基于RGBM编码的HDR纹理压缩这三个应用场景的原理。
然后在第三章阐述了所提出的可微DXT格式编解码模型,并在三个应用场景中进行实验,
之后在第四章中阐述了所提出的基于混合专家模型的编码选择器。
实验证明本研究所提出的方法相较于已有的仅解码模型,表现出了更好的结果,
基于混合专家模型的编码配置选择器很大程度提高了可微编解码器的计算效率,
使得所提出的方法相较固定配置,在运行速度层面的差距显著减小,同时可以获得更好的
编码质量。

\section{展望}

目前本文所提出的方法忽略了量化操作,这会与标准编解码器产生一定程度的差异,
最终降低与标准编解码器的相似程度。因此将可微量化过程加入可微编解码器可以进一步提高
与标准编解码器的近似程度,从而提升优化效果。
对于BC1至BC5与BC7这6种格式,由于导致量化
不可微的原因主要是均匀量化过程中使用的舍入函数,目前已有一些方法可在
优化过程中近似舍入函数的梯度。而对于非对称量化的BC6格式,由于量化过程中通过二进制的
重新解读实现了浮点数与整数的转化,这个过程的可微近似较难实现,因此BC6格式量化过程的可微近似还需要进一步探索。

尽管本文提出的可微纹理块压缩模型主要针对 DXT 格式,
但其中许多设计原则同样适用于其他纹理块压缩格式,
只需针对不同格式进行特定设计。例如,对于移动端平台中广泛使用的 ASTC 格式,
其涉及更多复杂问题,包括多达 3072 种分区配置、不同的块大小、
使用权重网格而非逐纹素权重、以及3D纹理。由于ASTC格式与DXT格式都具有分区和颜色端点
插值的特性,因此从这两个特点出发未来可以继续探索ASTC格式的可微近似。


