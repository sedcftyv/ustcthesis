% !TeX root = ../main.tex

\chapter{总结和展望}

\section{工作总结}

本文提出了一种新颖的可微块压缩模型,实现了DXT格式的可微编解码器,
可以作为DXT格式的可微近似整合到基于梯度的优化过程中并通过自动微分实现端到端优化,
从而降低对神经纹理进行块压缩引起的模型精度损失。
同时为了减少分区配置选择过程带来的显著计算开销,提出了基于混合专家模型
的分区配置选择器,可在几乎不损失编码质量的前提下,显著提升编码速度。

本文在第三章对DXT格式的可微块压缩模型进行了研究。
首先分析了DXT格式的主要特点与编码过程的本质问题。
由于DXT格式本质上都将每4\times4纹理块存储为颜色端点与权重,
不同格式区别仅在量化位数、分区、通道交换等配置上。
因此解决颜色端点与权重的计算问题即可解决DXT格式编码过程的主要问题。
而DXT格式的解码为根据纹素的权重对颜色端点进行的线性插值,这个过程天然可微。
然后提出了可微块压缩模型的整体框架,主要包括可微编码器、可微解码器和
配置选择器三部分。
之后阐述了可微编解码器的具体实现,
给出了编码过程利用主成分分析法(Principal Component Analysis, PCA)
解决编码问题以及解码过程根据权重对颜色端点进行线性插值的数学表达。
为了更接近DXT格式标准编解码器,对可微编解码后的
颜色端点与权重进行了模拟量化。
对于舍入函数导致的零梯度问题,使用神经网络量化领域常用的
直通估计器(Straight-Through Estimator, STE),使得
梯度得以正常进行反向传播。为了保证模拟量化的输入值在
0-1之间,量化前后还会对颜色端点与权重进行缩放。
之后设计了基于编码误差的配置选择器以解决多模式的BC6与BC7格式的配置选择问题。
为了使优化过程中神经纹理的值在LDR纹理的范围内,
设计了范围约束损失,使得浮点类型的神经纹理可以以
LDR格式进行存储。
最后在神经压缩模型中对所提出的方法进行了实验,
验证了支持编码配置动态选择的编解码器架构相较于
固定编码配置的仅解码器架构所具有的优势。
但性能测试中编码配置动态选择相较于固定配置显著增加的
运行时间很大程度影响了编解码器架构的实用性,
这引出了第四章中的工作。

本文在第四章对分区配置的选择问题进行了研究。
首先分析了多模式的BC6与BC7格式中,分区配置占所有
编码配置总数中绝大部分的特点。
确定了通过优化分区配置的选择从而减少计算开销的优化思路。
之后分析了第三章中编码配置选择器通过穷举所有编码配置
以计算编码误差,但由于最终只选择编码误差最小的编码配置
导致的大量无效计算的问题。
为解决这一问题,提出了一种基于混合专家模型的分区配置选择器。
将每个分区配置的编码误差的倒数定义为亲和度。
亲和度越高的分区配置优先级越高。
开始优化前通过穷举法初始化一个亲和度表,其中存储了
每4\times4块的分区配置对应的亲和度。
优化过程中通过被激活的分区配置更新亲和度表。
每次迭代过程中,将每4\times4块的所有分区配置
根据亲和度表中的亲和度进行TopK排序,
取排序后前几位的分区配置作为激活的稳定分区配置。
为了防止优化过程中退化为固定配置,
在剩余的分区配置中通过Gumbel-Max技巧
随机采样少量分区配置作为激活的随机分区配置。
最终将稳定分区配置与随机分区配置两部分作为
有效的分区配置。对于非分区配置由于数量较少
视为每次迭代都激活的稳定配置。
这种方式使用历史迭代的编码误差
近似当前迭代的编码误差,从而显著减少了
分区配置选择过程的计算量。
类似模拟退火算法,为了在优化初期增加对
编码配置的探索,为配置选择器也使用了
Gumbel-Max技巧,即在根据编码误差选择最佳配置时
加入随迭代次数衰减的Gumbel噪声。
使得优化初期可以充分探索不同的编码配置,
优化后期逐渐固定在使得编码误差最小的编码配置上,
从而避免陷入局部最优的编码配置中,找到全局更优
的编码配置。
最后在神经压缩模型、基于移动基分解的光照数据压缩模型、
HDR纹理的RGBM编码模型三个应用场景中进行了实验。
神经压缩模型的对比实验证明了使用分区配置选择器
与穷举所有配置相比实现了几乎一样,
甚至在一些测试数据上更好的优化效果。
同时性能实验表明使用分区配置选择器
相比穷举所有配置运行速度显著提高,
相比固定配置仅增加了少量运行时间。
对比实验与性能实验证明了第四章提出的分区配置选择器
的有效性。
基于移动基分解的光照数据压缩实验与HDR纹理的RGBM编码实验
中使用可微块压缩模型的方法相比不使用可微块压缩模型的
基线方法取得了明显更好的效果,证明了所提出方法的通用性。

\section{展望}

本文所提出可微块压缩模型扩展了传统纹理块压缩技术的使用场景。
随着神经网络技术在图形学领域的应用越来越广泛,
本文提出的方法有机会在更多基于神经网络的模型中使用。
本文的研究工作仍有改进和提升的空间,并可作为未来工作的研究方向:
\begin{itemize}
    \item BC6采用一种非对称的量化方式,由于量化过程中通过二进制的
    重新解读实现浮点数与整数的转化,
    这个过程的可微近似较难实现。
    因此本文所提出的方法忽略了BC6格式的颜色量化过程,
    这会与标准BC6编解码器产生一定程度的差异,
    导致实验过程中可微BC6的结果相比可微BC7差距比较明显。
    因此若能找到BC6格式颜色量化的可微近似方法,
    可以进一步提高可微BC6的优化效果。

    \item 可微块压缩模型的模拟量化过程中,
    本文使用直通估计器(Straight-Through Estimator, STE)解决
    舍入函数的不可微问题。
    STE使用恒等函数的梯度近似舍入函数梯度的过程可能会
    产生较大的误差,
    可以探索使用基于三角函数的阶梯型软舍入函数代替标准舍入函数,
    软舍入函数可以使梯度计算更接近标准舍入函数,
    可能能进一步提高可微块压缩模型的优化效果。

    \item   
    对于大量分区配置的选择问题,
    本文使用随机采样与基于亲和度TopK排序的方法选择
    出少量分区配置进行计算,
    未来可探索更好的编码配置选择方案,
    例如通过训练神经网络预测每个块的编码配置。


    \item   
    尽管本文提出的可微块压缩模型主要针对 DXT 格式,
    但其中许多设计原则同样适用于其他纹理块压缩格式,
    只需针对不同格式进行特定设计。例如,对于移动端平台中广泛使用的 ASTC 格式,
    其涉及更多复杂问题,包括多达 3072 种分区配置、不同的块大小、
    使用权重网格双线性上采样而非逐纹素的权重、以及3D纹理等。
    由于ASTC格式与DXT格式都具有分区和颜色端点插值的特性,
    因此从这两个特点出发未来可以继续探索ASTC格式的可微编解码器。


\end{itemize}




