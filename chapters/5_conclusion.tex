% !TeX root = ../main.tex

\chapter{总结和展望}

\section{工作总结}

本文提出了一种新颖的可微块压缩模型,实现了DXT格式的可微编解码器,
可以作为DXT格式的可微近似整合到基于梯度的优化过程中并通过自动微分实现端到端优化,
从而降低纹理块压缩技术引起的模型精度损失。
同时为了减少编码配置选择过程带来的显著计算开销,提出了基于混合专家模型
的分区配置选择器,可在几乎不损失编码质量的前提下,显著提升编码速度。

具体来讲,本文首先在第二章简要介绍了DXT纹理格式与混合专家模型的基本概念与相关工作,
以及神经压缩、基于移动基分解的光照数据压缩、基于RGBM编码的HDR纹理压缩这三个应用场景的原理。
然后在第三章阐述了所提出的可微块压缩模型,并在神经压缩模型中进行了实验,
之后在第四章中阐述了所提出的基于混合专家模型的分区配置选择器。
实验证明了本研究所提出的方法的有效性。
基于混合专家模型的分区配置选择器很大程度提高了可微编解码器的计算效率,
使得所提出的方法相较固定配置,在运行速度层面的差距显著减小,同时可以获得更好的
编码质量。

\section{展望}

目前本文所提出的方法忽略了BC6格式的颜色量化过程,
这会与标准BC6编解码器产生一定程度的差异,
导致实验过程中可微BC6的结果相比可微BC7有一定差距。
因此若能将BC6的颜色量化过程加入可微编解码器可以进一步提高
与标准BC6编解码器的近似程度,从而提升优化效果。
BC6采用一种非对称的量化方式,由于量化过程中通过二进制的
重新解读实现浮点数与整数的转化,
这个过程的可微近似较难实现,因此BC6格式的颜色量化过程的可微近似还需要进一步研究。

尽管本文提出的可微块压缩模型主要针对 DXT 格式,
但其中许多设计原则同样适用于其他纹理块压缩格式,
只需针对不同格式进行特定设计。例如,对于移动端平台中广泛使用的 ASTC 格式,
其涉及更多复杂问题,包括多达 3072 种分区配置、不同的块大小、
使用权重网格而非逐纹素权重、以及3D纹理。由于ASTC格式与DXT格式都具有分区和颜色端点
插值的特性,因此从这两个特点出发未来可以继续探索ASTC格式的可微近似。


