% !TeX root = ../main.tex

\chapter{总结和展望}

\section{工作总结}

本文提出了可微DXT格式编解码模型,可以集成到许多压缩方法中,
从而有效降低这些方法经过DXT格式二次压缩
造成的质量损失。同时对于编码配置选择带来的计算开销问题,提出了基于混合专家模型
的优化方法,可以在几乎不损失质量的前提下,将编码速度提升一个数量级。
具体来讲,本文首先在第二章简要介绍了DXT纹理格式与混合专家模型的基本概念与相关工作,
以及神经纹理压缩、基于移动基分解的光照数据压缩、基于RGBM编码的HDR纹理压缩这三个应用场景。
然后在第三章阐述了所提出的可微DXT格式编解码模型的具体内容,并在三个应用场景中进行实验,
之后在第四章中阐述了所提出的基于混合专家模型的编码选择器。
实验证明本研究所提出的方法相较于已有的仅解码模型,表现出了更好的结果,
基于混合专家模型的编码配置选择器很大程度提高了可微编解码器的计算效率,
使得所提出的方法相较于仅解码模型运行速度的差距显著减小,有效地增加了实用性。

\section{展望}

目前本文所提出的方法忽略了量化操作,这会与标准编解码器产生一定程度的差异,
最终降低与标准编解码器的相似程度,未来可以继续探索量化对可微编解码器的影响。

尽管本文提出的可微纹理块压缩模型主要针对 DXT 格式,
但其中许多设计原则同样适用于其他纹理块压缩格式,
只需针对不同格式进行特定设计。例如,对于移动端平台中广泛使用的 ASTC 格式,
其涉及更多复杂问题,包括多达 3072 种分区配置、不同的块大小、
使用权重网格而非逐纹素权重、以及3D纹理等。


