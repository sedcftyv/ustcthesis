% !TeX root = ../main.tex

\ustcsetup{
  keywords  = {纹理压缩},
  keywords* = {Texture Compression},
}

\begin{abstract}

  为了减少渲染过程中使用的大量纹理所占用的空间,同时保持高效的GPU硬件采样特性,
  纹理块压缩目前已经成为计算机图形学必不可少的组成部分。
  然而目前的纹理块压缩编解码器通常被设计为高效进行编解码,而无法与其他数据的压缩方式结合,
  进行端到端联合优化,同时这些工具只关注降低纹理本身的压缩误差。
  这造成了对于其他数据的压缩方式产生的纹理形式的中间数据,现有的纹理压缩工具
  直接进行压缩会产生较大的质量损失。
  
  针对上述问题,本文提出了一种新颖的可微纹理块压缩模型。
  该方法通过实现可微的编解码器,使得编解码流程可以整合到其他数据压缩模型的训练或生成过程当中。
  从而,相比于分阶段压缩,通过端到端的训练更好地挖掘出块压缩技术的优化空间。
  对于编码过程中的众多编码配置选择造成的计算开销,本文提出了一种基于混合专家模型的
  编码配置选择器,通过随机采样与基于top-k的选择策略,在几乎不影响编码质量的
  前提下,将编码速度提升了一个数量级。

  具体来说,本文的研究内容和主要贡献如下:
  
  (1)提出了一种可微的DXT格式纹理块压缩模型。基于DXT格式的定义,设计可微的
  编码器、解码器,以及配置选择器,从而实现DXT格式的可微编解码器。该方法在三个应用
  场景中与基线模型进行了对比,均展现出了优越的性能,从而验证了有效性。
  
  (2)在(1)的基础上提出了一种基于混合专家模型的编码配置选择器。该方法将编码配置
  根据编码误差通过top-k排序,分为始终激活的优先配置与通过随机采样激活的候选配置。
  只需使用数量较少的优先配置与候选配置,就能接近枚举所有配置的编码质量,同时显著地
  减少了计算开销。该方法同样在三个应用场景中与基线模型进行了对比,均能在保持质量的前提下,
  相较基线模型较大程度地提升速度,从而验证了有效性。

  % 摘要是论文内容的总结概括,应简要说明论文的研究目的、基本研究内容、研究方法或过程、结果和结论,突出论文的创新之处。
  % 摘要应具有独立性和自明性,即不用阅读全文,就能获得论文必要的信息。
  % 摘要中不宜使用公式、图表,不引用文献。

  % 摘要分中文和英文两种,中文在前,英文在后,内容及段落须相互呼应。博士论文中文摘要一般800~1000个汉字,硕士论文中文摘要一般600个汉字。
  % 英文摘要的篇幅参照中文摘要。

  % 论文的关键词,是为了文献标引工作从论文中选取出来用以表示全文主题内容信息的单词或术语。建议关键词数量不超过8个,每个关键词之间用分号间隔。

  % 英文摘要部分的标题为“ABSTRACT”。
  % 每个关键词第一个字母大写,关键词之间用半角逗号加空一格间隔,英文关键词与中文关键词须相互呼应。

\end{abstract}

\begin{abstract*}
  % The length of the English abstract should refer to that of the Chinese abstract.
  % The title of the English abstract is “ABSTRACT”.
  % The first letter of each keyword should be capitalized, and keywords should be separated by a halfwidth comma and a space.
  % The English keywords and Chinese keywords should correspond to each other.
\end{abstract*}
